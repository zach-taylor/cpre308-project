\section{Background}
\label{s:background} % This declares a label so you can reference the section elsewhere.

To understand why address space layout randomization is important, it is necessary to make sense of the kinds of attacks it helps mitigate. Buffer overflow attacks are the main class of attacks that ASLR tries to prevent. A buffer overflow happens when the data being entered into a buffer is larger than the allocated buffer space. The data typically then is placed in the adjacent memory locations to the buffer. It is usually unknown what is in those overwritten memory locations. An example of this bug can be seen in Tables \ref{t:fig1a} and \ref{t:fig1b}. In this example, a string of length eight is placed into a buffer of length six. Since the input data is longer than the buffer, the extra data is overflowed into the following memory locations. In this instance, the following memory locations make up another variable.

\begin{lstlisting}[caption=Code before Table \ref{t:fig1a}]
char buffer[6] = { };
unsigned short other = 0x07BB;
\end{lstlisting}

\begin{table}[h]
\centering % Make it center
\begin{tabular}{|l|l|l|l|l|l|l|l|l|}
\hline
\multicolumn{1}{|c|}{var} & buffer{[}0{]} & buffer{[}1{]} & buffer{[}2{]} & buffer{[}3{]} & buffer{[}4{]} & buffer{[}5{]} & other{[}0{]} & other{[}1{]} \\ \hline
hex                       & 00            & 00            & 00            & 00            & 00            & 00            & 07           & BB           \\ \hline
\end{tabular}
\caption{The buffers after initialization}
\label{t:fig1a}
\end{table}

\begin{lstlisting}[caption=Code before Table \ref{t:fig1b}]
strcpy(buffer, "8 lttrs");
\end{lstlisting}

\begin{table}[h]
\centering % Make it center
\begin{tabular}{|l|l|l|l|l|l|l|l|l|}
\hline
\multicolumn{1}{|c|}{var} & buffer{[}0{]} & buffer{[}1{]} & buffer{[}2{]} & buffer{[}3{]} & buffer{[}4{]} & buffer{[}5{]} & other{[}0{]} & other{[}1{]} \\ \hline
hex                       & 38            & 20            & 6C            & 74            & 74            & 72            & 73           & 00           \\ \hline
\end{tabular}
\caption{The buffers after overflow}
\label{t:fig1b}
\end{table}

This failure to check the bounds on the input causes a bug. A malicious attacker could potentially use bugs like these to overwrite memory in their favor. One common method to exploit these bugs is to overwrite the return address on the program's stack. This is called a stack buffer overflow. When an exploited function returns, it jumps to the address in the return address specified by the stack. The program will begin execution somewhere other than the intended place. There are many variations of the buffer overflow attack, but this demonstrates how severe these bugs are.

The idea behind ASLR was first implemented by Memco Software in 1999 \cite{yarom1999method}. ASLR first showed up in Linux as part of PaX project in 2003 \cite{paxdocs}. Since then, several other operating systems have implemented versions of ASLR, including Windows~\cite{msexploitmitigation} and Mac OS X \cite{applesecurity}. This demonstrates the importance of this security feature in modern operating systems.
